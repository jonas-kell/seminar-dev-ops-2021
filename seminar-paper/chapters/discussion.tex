\chapter{Diskussion: Dev-Sec-Ops}

Diese Arbeit stellt den schriftlichen Teil des Seminars \emph{Softwareentwicklung mit Dev(Sec)Ops} dar, das im Sommersemester 2021 im Zuge des B.Sc. Informatik an der Universität Augsburg angeboten wurde.

Die Arbeit setzt sich zusammen aus drei Teilen: Einem allgemeinen ersten Teil, indem das Thema \emph{Dev(Sec)Ops} (im folgenden vereinfachend oft nur \emph{DevOps}) ohne besondere Vorgaben in einem generellen Diskurs behandelt werden soll. Einem zweiten Teil, in dem die in dem Seminar behandelten Werkzeuge, Anwendungen und Methoden in einem möglichst funktionalen Gesamtsystem modelliert werden. Und einem dritten Teil, der an die Präsentation, bzw. den Vortrag \emph{Compliance: License-Checker} des Autors anknüpft, der ebenfalls im Zuge des Seminars gehalten wurde und einige der dort präsentierten Punkte nochmals schriftlich ausführt.

\section{Warum Dev-Sec-Ops}

Um zu verstehen, warum es in der modernen Softwareentwicklung nahezu unerlässlich geworden ist auf \emph{Dev(Sec)Ops} zurückzugreifen, muss zunächst der Begriff an sich verstanden werden \cite{forcepointWhatDevSecOpsDefined} \cite{redheadWasIstDevSecOps}


\section{Für wen ist Dev-Sec-Ops?}

Die kurze Antwort wird der Leser bereits erahnen können: \glqq Für JEDEN!!\grqq{}. Die ausführlichere, dafür aber auch zufriedenstellendere Antwort ist komplizierter. 

Nach 