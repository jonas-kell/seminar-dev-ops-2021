\chapter{Diskussion: Dev(Sec)Ops}

Diese Arbeit stellt den schriftlichen Teil des Seminars \emph{Softwareentwicklung mit Dev(Sec)Ops} dar, das im Sommersemester 2021 im Zuge des B.Sc. Informatik an der Universität Augsburg angeboten wurde.

Die Arbeit setzt sich zusammen aus drei Teilen: Einem allgemeinen ersten Teil, indem das Thema \emph{Dev(Sec)Ops} (im folgenden vereinfachend oft nur \emph{DevOps}) ohne besondere Vorgaben in einem generellen Diskurs behandelt werden soll. Einem zweiten Teil, in dem die im Seminar behandelten Werkzeuge, Anwendungen und Methoden in einem möglichst funktionalen Gesamtsystem modelliert werden. Und einem dritten Teil, der an die Präsentation, bzw. den Vortrag \emph{Compliance: License-Checker} des Autors anknüpft, der ebenfalls im Zuge des Seminars gehalten wurde und einige der dort präsentierten Punkte nochmals schriftlich ausführt.

\section{Warum Dev(Sec)Ops}

Um zu verstehen, warum es in der modernen Softwareentwicklung nahezu unerlässlich geworden ist auf \emph{Dev(Sec)Ops} zurückzugreifen, muss zunächst der Begriff an sich verstanden werden. \emph{Dev(Sec)Ops} ist eine Zusammensetzung aus drei Abkürzungen und steht für \emph{Development(Security)Operations}, also \emph{Entwicklung(Sicherheit)Betrieb} \cite{forcepointWhatDevSecOpsDefined}. 

Damit stellt DevOps ein Gerüst von Prinzipien dar, die ein Team bei der Entwicklung einer Anwendung über den ganzen Produktlebenszyklus (von der Entwicklung bis zum Betrieb) hinweg unterstützen. DevOps hat keine festgeschriebene Definition und ist keine Anleitung der man Schritt für Schritt folgen kann. Es ist vielmehr ein Sammelbegriff für verschiedene Denkweisen, Praktiken und Werkzeuge \cite{awsWasIstDevOps}.

Dabei haben alle der Methodiken gemein, dass durch sie eine Verkürzung der Reaktionszeit eines Entwicklerteams ermöglicht werden kann. 
Wie bereits erwähnt, erstreckt sich diese Suche nach Zeitersparnis über den gesamten Lebenszyklus einer Anwendung. So kann beispielsweise durch einen \emph{automatisierten Test} der bereits auf der Maschine des Entwickler ausgeführt werden kann, verhindert werden, dass ein Programmierfehler in die Produktionsumgebung vordringen kann. Dies wäre ein Beispiel für die \emph{Development} Stufe. Auf der anderen Seite lassen sich auch Verfahren betrachten, die der \emph{Operations} Stufe beigemessen werden können. Ein \emph{aktives Monitoring} einer Anwendung kann das Entwicklerteam sofort auf einen Ausfall eines Systems hinweisen. Im besten Fall können damit Fehler behoben werden, bevor sie beim Endnutzer zu Problemen führen. Ohne das Monitoring als DevOps Baustein, wird der Entwicklerstab im besten Fall erst herangezogen wenn die Menge an Beschwerden und Schadensmeldungen zu hoch wird, im schlechtesten Fall gar nicht. In keinem der letzten beiden Fälle ist dies für den Ruf eines Produktes zuträglich.

Durch diese hohe Reaktivität ist die DevOps Mentalität eng mit den Methodiken der \emph{agilen Entwicklung} verwandt \cite{haufe-lexwaregmbhcokgAgileMethodenDefinition}. In beiden Fällen ist es das Ziel möglichst \emph{agil}, also schnell und flexibel auf (sich verändernde) Anforderungen reagieren zu können.
Die agile Entwicklung stellt eine Sammlung von Organisationsformen dar, die ein flexibles und modernes Entwicklungsklima erzeugen. Sie zeichnen sich vor allem dadurch aus, dass das jedem Mitarbeiter individuell mehr Verantwortung und Vertrauen zukommt, die Teamgrößen kleiner werden, die Hierarchien flacher und die Release-Zeiten kürzer \cite{haufe-lexwaregmbhcokgAgileMethodenDefinition}. Dadurch wird die Produktivität des Teams gesteigert und es bleibt offen für Veränderungen. Eines der bekanntesten Beispiele für eine agile Methode ist \emph{Scrum}. Dieses fokussiert sich darauf, den Entwicklern Zeit zum ungestörten Arbeiten ohne Unterbrechungen zu geben und gleichzeitig die Produktivität und Abschlussrate hoch zu halten und das Ziel nicht aus den Augen zu verlieren \cite{froemlingAgileMethodenWas2021}.

Bleibt noch der dritte Teil der Definition, die \emph{Security}. Dabei ist die Idee, die IT-Sicherheit nicht als eigenständigen Abschnitt im Lebenszyklus des Projektes zu betrachten, sondern kollaborativ jeden mit in die gemeinsame Verantwortung zu ziehen, die Sicherheit \emph{während} jeder Phase mit in den Ablauf zu integrieren \cite{redheadWasIstDevSecOps}.

Letzter Begriff, der im Zusammenhang mit Dev(Sec)Ops einhergeht, ist der des \emph{CI/CD/CT}. Diese Reihe von Abkürzungen steht für \emph{Continuous Integration/Continuous Delivery/Continuous Testing} \cite{kinsbrunerHowMakeCI2018}. Damit bilden diese drei Komponenten die Teile der sogenannten \emph{Pipeline}, eine automatisiert ausführbare Sammlung von Werkzeugen und Skripten um den Code des Projektes bei Änderungen zu \emph{integrieren} (z.B. durch automatisierte Versionskontrolle, Buildprozesse und Coding-Style Überprüfungen), in die Staging- oder Produktionsumgebung \emph{auszurollen} (z.B. durch automatisches/inkrementelles Patchen auf dem Produktionsserver, kompilieren für verschiedene Betriebssysteme und Monitoring aktiver Applikationen) und in jedem Schritt des Prozesses möglichst kontinuierlich zu \emph{testen} (z.B. durch Unittests, Komponententests oder Sicherheitstests (Pentests)). 

Damit stellt das Dev(Sec)Ops einen Sammelbegriff für eine \emph{Entwicklungs-Kultur} \cite{kolblSoftwareentwicklungMitDev2021} dar, unter deren Schirm Prozesse, Technologien und Personen zusammenkommen. Dabei liegt diesen allen ein \emph{feedbackgesteuertes, reaktives} Modell zugrunde und damit kann dies als die zentrale Idee beziehungsweise Eigenschaft gesehen werden.

\section{Für wen ist Dev(Sec)Ops?}

Im folgenden Abschnitt soll eine grundlegende Frage beantwortet werden: \glqq Für wen ist die DevOps Kultur?\grqq{}. 
Ist es nur eine neuartige Erscheinung, oder etwas, dessen Prinzipien sich nur die marktführenden Riesenkonzerne zu Nutze machen können?

Die kurze Antwort wird der Leser bereits erahnen können: \glqq Für JEDEN!\grqq{}. Die ausführlichere, dafür aber auch zufriedenstellendere Antwort ist komplizierter. 

Warum die DevOps Kultur für jedes Entwicklerteam geeignet ist, hat zunächst eine ganz grundlegenden Ursache, die im ersten Abschnitt dargelegt wurde: Es gibt keine festen Regeln oder Vorgaben. Möchte man DevOps Kultur in sein Unternehmen oder Projekt einfließen lassen, so muss man keine Standards erfüllen, keine Mindestmaße erreichen. 
Man kann soviel oder wenig wie man möchte inkrementell einführen.
Zugegeben, manche Prozesse und Technologien werden in Kombination mächtiger als alleine, aber was nicht ist kann ja noch werden.
Und eine Umgebung, die bereits Vorteile durch einzelne Elemente eines DevOps Prozess genießt, ist in den meisten Fällen offener für weitere Elemente. 

Gerade die agilen Methoden \emph{Kanban} oder \emph{Scrum} \cite{froemlingAgileMethodenWas2021} gehören in den heutigen Zeiten zum Quasi-Standard, was das Projektmanagement angeht. Kaum ein neues Projekt kann heutzutage noch strikt nach dem Wasserfallmodell durchgeführt werden \cite{tutorialspointSDLCWaterfallModel}. Zu unvorhersehbar sind die Entwicklungen des Software-Marktes und zu gering die verfügbare Zeit für Planung.
Ideen verwerfen und iterativ testen zu können triumphiert in den meisten Anwendungsfällen, da es keine \glqq perfekten\grqq{} Lösungen gibt.
Durch die hohe Zahl an Frameworks und Drittanbieter-Paketen gibt es oft viele verschiedene Lösungsstrategien für ein Problem.
Oft ist es also gar nicht möglich, im Voraus die \glqq ideale\grqq{} Lösung zu kennen oder gar zu planen. An einem iterativen Entwicklungsansatz führt also kein Weg vorbei.

Neben den Punkten \emph{inkrementeller Adaption} und \emph{agiler Methoden} stellt allerdings der dritte hier aufgeführte den wichtigsten dar: die Werkzeuge.
CI/CD Werkzeuge können bereits mit kleinem Aufwand große Kosten- und Zeitersparnis verursachen. 
Betrachte man das einfache Beispiel: Automatisches Ausrollen in die Produktionsumgebung \cite{sonBeginnerGuideBuilding2019}. 
Diese Aufgabe ist oft simpel und mit wenigen Kommandozeilen-Befehlen erledigt. Allerdings ist die Aufgabe repetitiv (Ausrollen bei jedem Release, auf mehreren Servern), mitunter Zeitaufwändig (Lange Build-Zeiten / Downloads) und unsicher (wenn jeder Zugriff auf die Produktionsumgebung hat, ist dies ein Sicherheitsrisiko).
All dies macht den Prozess fehleranfällig bei manueller Durchführung, jedoch ideal für das Ausführen durch eine Maschine. 
Der initiale Mehraufwand durch die Konfiguration der Werkzeuge wird in der Regel bereits nach wenigen Zyklen ausgeglichen sein. 

Es lässt sich also zusammenfassend sagen, dass jedes Element aus dem Bereich der DevOps-Kultur ein Team bereichern kann, ohne es dabei einzuschränken.
Da es also praktisch nur Vorteile mit sich bringt, eignet sich DevOps im Prinzip für jeden.
